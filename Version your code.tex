\subsection{Version your code}

The first step is to learn how to version your own code. In this tutorial, we will run Git from the command line of the Unix shell. Thus we expect readers are already comfortable with navigating a filesystem and running basic commands in such an environment. There are many graphical user interfaces (GUIs) available for running Git (Table x), which we encourage you to explore, but learning to use Git on the command line is necessary for performing more advanced operations and using Git on a remote server.

To start versioning your code with Git, navigate to your newly created or existing project directory. Start tracking your code by running the command \verb|git init|, which \textit{init}iates a new Git repository. Next you'll need to give Git some information about yourself, since it records who makes each change to the file(s). Provide your name and email by running the following lines \verb|git config --global user.name "First Last"| and \verb|git config --global user.email "user@domain"|, respectively. Now you're ready to start tracking your code.
