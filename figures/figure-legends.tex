\subsection{Figure Legends}

\textbf{Figure 1. The git add/commit process.}
To store a snapshot of changes in your repository, first \verb|git add| any files to the staging area you wish to commit (for example, you've updated the \verb|process.sh| file).
Second, type \verb|git commit| with a message. Only files added to the staging area will be committed.
All past commits are located in the hidden \verb|.git| directory in your repository.

\textbf{Figure 2. Working with both a local and remote repository as a single user.}
A) On your local machine you have a repository that you've committed to (commit d75es).
B) On GitHub, you created a new repository called \verb|thesis". This repository is currently empty and not linked to the repo on your local machine.
C) The command \verb|git remote add| connects your local repository to your remote repository.
The remote repository is still empty, however, because we have not pushed any content to it.
D) All the local commits are sent to the remote repository by using the command \verb|git push".
Only files that have been committed will appear in the remote repository.
E) Several more rounds of updating scripts and committing are done on the local computer (commit f658t and then commit xv871).
These commits have not yet been pushed to the remote repository, so only the previously pushed commit is in the remote repo (commit d75es).
F) To sync the local and remote repositories, \verb|git push| the two new commits to the remote repository.
The local and remote repositories now contain the same files and commit histories.

\textbf{Figure 3. Contributing to Open Source Projects.}
You found an error in the README for a cool project hosted on Github and you would like to fix it.
A) The cool\_project repository exists on Github (along with the commit history of the project), but you would like to edit the README on your computer.
B) First, you fork the cool\_project repository into your account on Github, bringing the entire commit history of the project along with the code.
C) To create a copy of the repository on your computer, you \verb|git clone| the repository from your Github account.
D) You make a change to the README, save the file, add the file to the staging area, and commit (creating commit 09pr4).
This commit is only on your local computer and your remote cool\_project repository and the original remote cool\_project repository do not have this commit.
E) You sync your local and remote repositories by using \verb|git push".
The original cool\_project still does not have your update to the README file.
F) To suggest the change in the README to the original cool\_project team, submit a pull request via Github of your commits.
If the owners of the cool\_project repository like your change, they will accept the pull request and your changes will be incorporated into the project.

\textbf{S1 Figure. Collaboration using Github.}
A) You have a repository on your computer and on github that you have already connected using \verb|git remote add", but you have not pushed any content to the remote repo yet.
B) You push your local commits to your remote repository using \verb|git push| (both your local and your remote have commit d75es).
C) Your collaborator would like to help you analyze your data.
They make a copy of your remote repository into their github account using the "fork" option in github.
All three repositories have the same version history (commit d75es).
D) Your collaborators want a copy of the repository on their local computer so they can make edits.
They \verb|git clone| the repository from their remote github account.
E) Your collaborator edits a file and commits a change to their local repository (commit t957s).
Changes are not automatically shared across all repositories, therefore your local, your remote, and your collaborators remote only have commit d75es.
F) Your collaborator syncs their local repo with their remote repo by pushing.
Now both your collaborator's local and remote repositories have two commits (commit d75es and commit t957s), while your local and remote only have one (commit d75es).
G) Unaware of what your collaborator is doing, you continue to work on your code and you make a new commit locally (commit f658t).
H) Your collaborator wants you to incorporate their changes into your code, so from Github they issue a Pull Request.
The changes are acceptable, so you confirm the Pull Request and merge the changes into your remote repository.
Your remote repository is ahead of your local repository by one commit: the commit your collaborators made (commit t957s) and in addition your local repository is ahead of your remote repository by one commit that was made before the Pull Request was issued by your collaborators (commit f658t).
I) To incorporate the new commit from your collaborator into your local repository, you \verb|git pull| commits from your remote to your local repository.
Your local now has the commit from your collaborator incorporated, but is still ahead of your remote by one commit.
You could \verb|git push| your changes to sync your remote and local repositories.
