\textbf{Figure 3. Collaboration using Github.} A) You have a repository on your computer and on github that you have already connected using \verb|git remote add|, but you have not pushed any content to the remote repo yet. B) You push your local commits to your remote repository using \verb|git push| (both your local and your remote have commit ID d75es). C) Your collaborator would like to help you analyze your data. They make a copy of your remote repository into their github account using the "fork" option in github. All three repositories have the same version history (commit ID d75es). D) Your collaborators want a copy of the repository on their local computer so they can make edits. They \verb|git clone| the repository from their remote github account. E) Your collaborator edits a file and commits a change to their local repository (commit ID t957s). Changes are not automatically shared across all repositories, therefore your local, your remote, and your collaborators remote only have commit ID d75es. F) Your collaborator syncs their local repo with their remote repo by pushing. Now both your collaborator's local and remote repositories have two commits (commit ID d75es and commit ID t957s), while your local and remote only have one (commit ID d75es). G) Unaware of what your collaborator is doing, you continue to work on your code and you make a new commit locally (commit ID f658t). H) Your collaborator wants you to incorporate their changes into your code, so from Github they issue a Pull Request. The changes are acceptable, so you confirm the Pull Request and merge the changes into your remote repository. Your remote repository is ahead of your local repository by one commit: the commit your collaborators made (commit ID t957s) and in addition your local repository is ahead of your remote repository by one commit that was made before the Pull Request was issued by your collaborators (commit ID f658t). I) To incorporate the new commit from your collaborator into your local repository, you \verb|git pull| commits from your remote to your local repository. Your local now has the commit from your collaborator incorporated, but is still ahead of your remote by one commit. You could \verb|git push| your changes to sync your remote and local repositories. 